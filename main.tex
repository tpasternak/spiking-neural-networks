\documentclass{beamer}
\usepackage{polski}
\usepackage[utf8]{inputenc}
\usepackage{pgfplotstable}
\usepackage{pgfplots}
\usetikzlibrary{calc}
\pgfplotsset{compat=1.5}
\usetheme{Warsaw}

  \setbeamercovered{transparent}

\author {Karol Olko, Tomasz Pasternak}
\title {Spiking Neural Networks}
\begin{document}
\frame{\titlepage}
\frame{\tableofcontents}
\section{Wprowadzenie}

\begin{frame}
  \begin{block}{Podejście tradycyjne}
    \begin{itemize}
    \item Bierzemy pod uwagę jedynie częstotliwość wysyłania impulsów jako nośnik informacji między neuronami
    \item Duże straty informacji
    \end{itemize}
  \end{block}
\end{frame}
\begin{frame}
  \begin{block}{Podejście impulsowe}
    \begin{itemize}
    \item<1-> Czas wyjścia każdego impulsu jest brany pod uwagę
    \item<2-> O stanie wyjściowym neuronu może decydować kolejność impulsów
    \item<3-> Wierniejsze odwzorowanie biologicznych NN
    \item<4-> Wymagana większa moc obliczeniowa
      \end{itemize}
  \end{block}
\end{frame}
\begin{frame}
  \begin{center}
    \begin{tikzpicture}
      \begin{axis}[name = neuron1, title=Neuron 1, height=3cm, width=10cm]
        \addplot[blue] table {neuron1.dat};
      \end{axis}
      \begin{axis}[name = neuron2, title=Neuron 2, at={($(neuron1.south west)-(0,2cm)$)}, height=3cm, width=10cm, anchor=west]
        \addplot[blue] table {neuron2.dat};
      \end{axis}
    \end{tikzpicture}
  \end{center}
  Neuron posiadający neurony wejściowe wysyłające takie sygnały może zachować się inaczej, gdy przyjmiemy model impulsowy.
\end{frame}
\section{Modele impulsowe}
\section{Kodowanie informacji}
\section{Uczenie}
\end{document}
