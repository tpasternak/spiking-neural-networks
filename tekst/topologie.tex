Posiadając model neuronów i synaps jesteśmy w stanie zdefiniować impulsową sieć neuronową. Typowo, SNN jest traktowana jak skończony graf skierowany (V,E), z V reprezentującymi zbiór neuronów i E - zbiór synaps. Wystrzał neuronów wejściowych jest determinowany na zewnątrz sieci - są one traktowane jako wejście impulsowej sieci neuronowej.

Topologie impulsowych sieci neuronowych są dzielone na trzy kategorie:
1. Sieci w przód (ang. \textit{feedforward}) - przepływ danych z wejść do wyjść jest ściśle jednokierunkowy. Przetwarzanie danych może odbywać w wielu warstwach neuronów, ale nie występują sprzężenia zwrotne. W biologicznych sieciach neuronowych sieci w przód znajdują się głównie w obszarach peryferyjnych. Podobnie, w SNN topologie w przód znajdują głównie zastosowanie w odwzorowywaniu niskopoziomowych zmysłów, takich jak wzrok, węch, dotyk. Sieci w przód są również rozważane w kontekście synchronizacji impulsów, czy przy powiazaniu wzorców przestrzenno-czasowych impulsów.

2. Sieci rekurencyjne - tutaj indywidualne neurony lub populacje tychże komunikują się ze sobą za pomocą wzajemnych połączeń. Wynikiem występowania sprzężeń zwrotnych jest występowanie stanu sieci, które pozwala na traktowanie jej jako system dynamiczny. W konsekwencji sieci rekurencyjne charakteryzują się bogatszą dynamiką i większymi możliwościami obliczeniowymi niż sieci w przód. Niestety, są również trudniejsze w kontroli i uczeniu. Sieci rekurencyjnych używano przy modelowaniu pamięci asocjacyjnej bądź roboczej. Sieci impulsowej z połączeniami rekurencyjnymi były również wykorzystane w analizie zjawisk obserwowanych w mózgu, mających nietrywialny przebieg dynamiczny, wynikający z wzajemnych połączeń między neuronami. 

3. Sieci hybrydowe - grupa ta zawiera sieci, w których niektóre subpopulacje mogą być ściśle w przód, a inne - rekurencyjne. Interakcje między subpopulacjami mogą być jednokierunkowe bądź wzajemne. O ile można wyobrazić sobie wiele potencjalnych architektur takich sieci, skupiono się tu na dwóch najbardziej popularnych i zbadanych klas hybrydowych sieci neuronowych:
\begin{itemize}
\item \textit{Synfire chain} (ang. łańcuch pobudzeń) - ludzka nauka często odbywa się poprzez asocjację dwóch wydarzeń, lub łączenia sygnału i następującej po nim reakcji w pewną relację. Wydarzenia są często odległe w czasie, ale mimo wszystko ludzie są w stanie połączyć je ze sobą, pozwalając na dokładną predykcję odpowiedniego czasu na wykonanie określonej akcji. Łańcuch pobudzeń jest uważany za prawdopodobny mechanizm reprezentujący takie relacje między opóźnionymi względem siebie wydarzeniami. Cechuje się on wielowarstwową architekturą (łańcuch), w którym aktywność impulsów może być propagowana jako synchroniczna fala wystrzałów neuronów ("paczka" impulsów) z jednej warstwy (subpopulacji) łańcucha do kolejnych. Definicja ta sugeruje topologię w przód, niemniej niektóre subpopulacje mogą posiadać wzajemne połącznia.
\item \textit{Obliczenia rezerwuarowe} - jest to obliczeniowa idea, wykorzystująca korzystne cechy sieci rekurencyjnych przy jednoczesnym ominięciu trudności związanych z ich treningiem. W typowej implementacji sieć rezerwuarowa zawiera stałą strukturę rekurencyjną (rezerwuar) i zbiór neuronów wyjściowych, zwanych odczytowymi. Zwykle neurony, odczytowe posiadają wyłącznie jednokierunkowe impulsy z rezerwuaru. Trening sieci polega na uczeniu połączeń między rezerwuarem a warstwą wyjściową. Takie podejście znacznie upraszcza uczenie tych sieci. 
Rezerwuar może być postrzegany jako struktura dokonująca mapowania wejść na wielowymiarowy wektor aktywności neuronów wchodzących w skład sieci. Każdy element tego wektora odzwierciedla wpływ, który poszczególne neurony mogą mieć na elementy wyjściowe. Struktura połączeniowa wewnątrz rezerwuaru jest zwykle losowa i ustalona (niezmienna w trakcie działania). Stabilny stan wewnętrzny rezerwuaru nie jest konieczny do produkowania stabilnych odpowiedzi sieci, gdyż przejściowe stany wewnętrzne mogą być rzutowane na stabilne wyjścia, wykorzystując wysoką wymiarowość systemu dynamicznego. Ponadto, stan rezerwuaru i przejścia pomiędzy nimi nie muszą być dostosowywane do konkretnych zadań. Oznacza to, że ta sama, dostatecznie duży, generyczny rezerwuar może być wykorzystywany przy zadaniach obliczeniowych różnego rodzaju. Sieci rezerwuarowe były z sukcesem wykorzystywane do takich zadań jak rozpoznawanie mowy ludzkiej, predykcji ruchu czy jego kontroli.
\end{itemize}